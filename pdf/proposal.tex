% You should title the file with a .tex extension (hw1.tex, for example)
\documentclass[11pt]{article}

\usepackage{enumerate}
\usepackage{textcomp}
\usepackage{fancyhdr}
\usepackage{amsmath, amsfonts, amsthm, amssymb, mathtools}
\usepackage{hyperref}

\oddsidemargin0cm
\topmargin-2cm     %I recommend adding these three lines to increase the
\textwidth16.5cm   %amount of usable space on the page (and save trees)
\textheight23.5cm

\newcommand{\question}[2] {\vspace{.25in} \hrule\vspace{0.5em}
\noindent{\bf #1: #2} \vspace{0.5em}
\hrule \vspace{.10in}}
\renewcommand{\part}[1] {\vspace{.10in} {\bf (#1)}}

\newcommand{\myname}{Brandon Hoane}
\newcommand{\myandrew}{bhoane@andrew.cmu.edu}

\setlength{\parindent}{0pt}
\setlength{\parskip}{5pt plus 1pt}

\pagestyle{fancyplain}
\lhead{\fancyplain{}{\textbf{Project Proposal}}}      % Note the different brackets!
\rhead{\fancyplain{}{\myname\\ \myandrew}}
\chead{\fancyplain{}{CDM }}

\begin{document}

\medskip                        % Skip a "medium" amount of space
                                % (latex determines what medium is)
                                % Also try: \bigskip, \littleskip

\thispagestyle{plain}
\begin{center}                  % Center the following lines
{\Large CDM Project Proposal} \\
\myname \\
\myandrew \\
\today \\
\end{center}

\subsection*{Proposal}
I will be implementing Safra's Determinization Algorithm for
B\"uchi Automata. Given a B\"uchi Automaton, we want to output a
Rabin Automata which recognizes the same $\omega$-language.

\subsection*{Goals}
\begin{itemize}
\item Correctly and concisely implement Safra's algorithm.
\item Convert examples in a reasonable amount of time.
\end{itemize}

\subsection*{Tools}
I'm using C++ implement Safra's algorithm.
Both automata and Safra trees will be represented using the Boost Graph Library.
The BGL makes it simple to swap out the internal representation of the graphs,
which should allow for easier optimization.
Furthermore, it provides hooks into Graphviz, which will allow
me to visualize the inputs and results.

\subsection*{Concerns}
\begin{itemize}
\item How can I show that conversions are performed correctly?
  It seems difficult to show that two $\omega$-automata are equivalent.
\item
  We know that we can determine acceptance on
  constant, periodic, and ultimately periodic inputs.
  It would be nice to implement these algorithms to
  demonstrate correctness, however this seems a little
  outside the scope of this project.
  Perhaps as a stretch goal.
\end{itemize}

\subsection*{Links}
\begin{itemize}
\item \href{http://www.boost.org/doc/libs/1_57_0/libs/graph/doc/}
  {Boost Graph Library}
\item \href{http://www.graphviz.org/}{Graphviz}
\end{itemize}

\end{document}
